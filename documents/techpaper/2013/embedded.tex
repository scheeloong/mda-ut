\section{Embedded System}

The embedded system was designed to link the submarine's computer with all
electronic peripherals, with a high-level interface to access groups of peripherals.
The embedded system consists of a DE0-Nano FPGA, synthesizing custom circuitry to
communicate with each peripheral and a soft processor to communicate with the computer. The FPGA was chosen
for its quantity of general-purpose I/O pins, allowing it to manage all electronic peripherals,
and its compact size of 3 x 2 inches. The FPGA is the only programmable element in the submarine
besides the computer, reducing the amount of communication between programmable elements.

\subsection{FPGA Soft Processor}

The FPGA Soft Processor exposes high-level commands to the computer to access each group
of electronic peripherals individually. The five motor drivers are controlled as a group
by the soft processor's yaw, pitch, roll and depth PID controllers. The computer interface can set target
speed, yaw, and depth, which are inputs to the PID controllers. Speed is implemented as an offset
because the submarine does not have velocity measurement. The PID controllers are implemented on the
soft processor because they need immediate access to the motor drivers and attitude readings.
The IMU and depth sensor provide attitude readings and are used by the PID controllers and the
computer interface. Power management continually monitors each
voltage rail, and will power off the submarine if any voltage is invalid. The high-level commands are
text-based, allowing manual control to test each peripheral individually.
Future plans for sonar processing will be implemented on the FPGA, using the soft processor and
built-in DSP elements.

\subsection{FPGA Hardware}

The FPGA also synthesizes custom hardware to communicate with peripherals through general-purpose pins.
This flexibility can support any protocol, including SPI and UART, to communicate with peripherals.
Each peripheral protocol is a separate module, memory-mapped to the soft processor across Altera's
Avalon Bus. The hardware and soft processor run at 50 MHz.
